\chapter{Introduction to Python}


In this chapter, we will examine the basic facilities of the Python programming language, and go on to the more advanced features in the next.~\footnote{This material has been adapted from Web Programming in Python by Thiruvathukal, Christopher, and Shafaee, for which Thiruvathukal was able to revert rights to himself and his original co-authors. Hereafter it will be released under the open source licensing terms of this book in progress.}

\section{Characteristics of Python}

Here is a high-level look at the features of Python, for those some previous background in programming.

\paragraph{Scripting}
Python is a scripting language. This loosely means that you
do not have to compile Python programs; Python can execute them
directly. In fact, you can type lines directly into the Python interpreter~\footnote{also known as a REPL, for read, evaluate, and print loop}
and have it execute them interactively. Python does compile programs
into code it interprets, but it will handle the compilation itself, without
troubling you.

\paragraph{Nondeclarative}
Python programs consist of executable statements.
Python doesn't have declarations. In most languages, you declare a function;
in Python, you execute a statement that creates a function object
and assigns it to a variable. In other languages, you call a function by
giving its name and an argument list. The Python call looks exactly the
same, but instead of the name of the function, you specify the name of
the variable containing the function as its value.

\section{Stay Tuned}

More to come. I am converting this text from FrameMaker. This was done via Google Keep and its OCR to grab text from any image.

